\documentclass{article}
\usepackage{amsfonts}
\usepackage[utf8]{inputenc}
\usepackage[T1]{fontenc}
\usepackage{float}
\usepackage{bbm}
\usepackage[round]{natbib}
\usepackage{hyperref}
\usepackage{graphicx} % Required for inserting images
\usepackage{tikz}
\usepackage{parskip} % Add this to remove paragraph indentation
\usetikzlibrary{shapes,arrows,positioning,fit}
\usepackage{listings}
\usepackage{amsthm}  % Add this package for theorem-like environments
\usepackage{xcolor}
\usepackage{amsmath}
\usepackage{mathabx}
\usepackage{subcaption}  % For subfigures
\bibliographystyle{plain}

% Define theorem-like environments
\theoremstyle{definition}
\newtheorem{definition}{Definition}
\newtheorem{remark}{Remark}


\begin{document}



\section{MergeSort}
Antag at \textsc{MergeSort}$(A,p,r)$, implementeret som i CLRS sektion 2.3, bliver kaldt på 21 
elementer (dvs. $r-p+1=21$). Hvor mange kald bliver der totalt lavet til \textsc{Merge-Sort}? 
Hvad er antallet af kald generelt, når input har $n$ elementer? Argumentér for dine svar.

For et array med længde 21 kan vi beregne det præcise antal kald til MergeSort således:

\begin{enumerate}
    \item Først beregner vi dybden af rekursionstræet:
        \[ d = \lceil \log_2(21) \rceil =  \rceil = 5 \]
    
    \item På niveau $i$ i rekursionstræet har vi $2^i$ kald, hvor $i \in \{0,1,...,d-1\}$
    
    \item Det totale antal kald er summen over alle niveauer. 
    Vi observerer at dette er en geometrisk serie med første led $a=1$ og kvotient $r=2$. For en geometrisk serie gælder generelt at $\sum_{i=0}^{n-1} ar^i = a\frac{r^n - 1}{r-1}$, som med vores værdier kan omskrives:
        \[ \sum_{i=0}^{d-1} 2^i = \frac{2^d - 1}{2-1} = 2^d - 1 = 2^5 - 1 = 31 \text{ kald} \]
\end{enumerate}

For et generelt input af størrelse $n$ får vi:
\begin{itemize}
    \item Dybde: $d = \lceil \log_2(n) \rceil$
    \item Totalt antal kald: $2^{\lceil \log_2(n) \rceil} - 1$
    \item Dette giver præcis $2^{\lceil \log_2(n) \rceil} - 1 = \Theta(n)$ kald
\end{itemize}

\section{Rekursionsligninger}
Hvilke af disse rekursionsligninger har løsningen $T(n) = \Theta(n^2)$? Antag at $T(n) = 1$ for 
$n \leq 1$. Vælg ét eller flere korrekte svar og beskriv hvordan du kom frem til dem (både 
positive og negative svar).

\begin{enumerate}
    \item $T(n) = 4T(\lfloor n/2 \rfloor) + n\lg n$
    
    Vi kan bruge Master Theorem til at løse denne rekursionsligning.
    
    Lad os identificere parametrene:
    \begin{itemize}
        \item $a = 4$ (antal delproblemer)
        \item $b = 2$ (reduktionsfaktor)
        \item $f(n) = n\lg n$ (kombinationstid)
    \end{itemize}
    
    Vi sammenligner $n^{\log_b a} = n^{\log_2 4} = n^2$ med $f(n) = n\lg n$:
    
    Da $n^2$ er asymptotisk større end $n\lg n$, og
    \[ n^2 = \Omega(n\lg n) \]
    falder vi ind under tilfælde 1 af Master Theorem.
    
    Derfor er løsningen:
    \[ T(n) = \Theta(n^{\log_2 4}) = \Theta(n^2) \]
    
    Dette bekræfter at $T(n) = \Theta(n^2)$ er en korrekt løsning for denne rekursionsligning.
    
    \item $T(n) = 4T(\lfloor n/2 \rfloor) + n^2$
    
    Anvendelse af Master Theorem:
    \begin{itemize}
        \item $a = 4$ (antal delproblemer)
        \item $b = 2$ (reduktionsfaktor)
        \item $\log_b a = \log_2 4 = 2$
        \item $f(n) = n^2 = \Theta(n^{\log_b a} \log^k n) = \Theta(n^2 \log^0 n)$
    \end{itemize}
    
    Dette falder under \textbf{Tilfælde 2} af Master Theorem med $k = 0$:
    \[ T(n) = \Theta(n^{\log_b a} \log^{k+1} n) = \Theta(n^2 \log n) \]
    
    Konklusion: Denne rekursionsligning har \textit{ikke} løsningen $\Theta(n^2)$.
    
    \item $T(n) = 2T(\lfloor n/4 \rfloor) + n^2$
    korrekt.vi bruger master theorem. 

    \begin{itemize}
        \item $a = 2$ (antal delproblemer)
        \item $b = 4$ (reduktionsfaktor)
        \item $f(n) = n^2$ 
    \end{itemize}

    vi er i case 3 da
    \begin{enumerate}
        \item $n^2 = \Omega(n^{log_4 2+\epsilon})$ for $\epsilon < 14$.
        \item $2*(n/4)^2 \leq c*n^2$ for $\frac{2}{16} < c < 1$
    \end{enumerate}


    \item $T(n) = 9T(\lfloor n/3 \rfloor) + n^3$
    forkert
    Vi bruger Master Theorem.
    \begin{enumerate}
        \item $a=9$
        \item $b=3$
        \item $f(n) = n^3$
    \end{enumerate}
    vi er i case 2 Da
    $f(n)=n^3=\Theta(n^3 * lg^{k}*n)$ for k = 0
    dermed er $T(n)=n^2*lg^{k}*n=n^2*1=n^2$

    \item $T(n) = T(n-1) + n^2$
    Forkert. Korrekt løsning er $\Theta(n^3)$. 
    Vi viser det ved iteration:
    
    Udvikling af rekursionen:
    \begin{align*}
    T(n) &= T(n-1) + n^2 \\
         &= T(n-2) + (n-1)^2 + n^2 \\
         &\ \ \vdots \\
         &= T(1) + \sum_{k=2}^n k^2 \\
         &= \Theta(1) + \frac{n(n+1)(2n+1)}{6} - 1 \\
         &= \Theta(n^3)
    \end{align*}

    \item $T(n) = T(n-1) + n$
    korrekt
    Udvikling af rekursionen:
    \begin{align*}
    T(n) &= T(n-1) + n \\
         &= T(n-2) + (n-1) + n \\
         &\ \ \vdots \\
         &= T(1) + \sum_{k=2}^n k \\
         &= \Theta(1) + \frac{n(n+1)}{2} - 1 \\
         &= \Theta(n^2)
    \end{align*}
\end{enumerate}

\section{Del og hersk}
Antag at du har en rekursiv algoritme $A$, og lad $n$ betegne størrelsen på algoritmens 
input $X$. Hvis $n=1$ beregnes resultatet $A(X)$ i konstant tid. For $n > 1$ bruger algoritmen 
først $O(n\lg n)$ tid på at beregne fire inputs $X_1, X_2, X_3, X_4$, hver af størrelse $\lfloor n/2 \rfloor$, og 
$A(X_1), A(X_2), A(X_3), A(X_4)$ beregnes rekursivt. Endelig kombineres de rekursive svar til 
et output $A(X)$ i tid $O(n^2)$. Skriv en rekursionsligning der beskriver en øvre grænse 
på køretiden $T(n)$ af $A$ på input $X$ og find en løsning ved hjælp af master theorem. 
Argumentér for dit svar.

rekursionsligningen kan skrives som

\[
T(n) = \begin{cases}
O(1) & \text{hvis } n = 1 \\
4T(n/2) + O(n\lg n) + O(n^2) & \text{hvis } n > 1
\end{cases}
\]

Da $O(n^2)$ dominerer $O(n\lg n)$, kan vi forenkle dette til:
\[
T(n) = 4T(n/2) + O(n^2)
\]

Vi bruger Master Theorem:
\begin{itemize}
    \item $a = 4$ (antal delproblemer)
    \item $b = 2$ (reduktionsfaktor)
    \item $f(n) = O(n^2)$ (kombinationstid)
\end{itemize}

Vi sammenligner $n^{\log_b a} = n^{\log_2 4} = n^2$ med $f(n) = n^2$:

Da $n^2 = \Theta(n^2)*lg^{k} n = \Theta(n^2)$, for $k = 0$, falder vi ind under tilfælde 2 af Master Theorem.
Derfor er løsningen:
\[ T(n) = \Theta(n^2 \lg n) \]

\section{Køretid}
Betragt følgende kode i pseudo-kode notationen fra CLRS:

\begin{verbatim}
Loops(n)
1  r = 0
2  for i = 1 to n/2
3      for j = n/2 - i to n
4          r = r + j
5  return r
\end{verbatim}

Vi antager at $n/2$ er et heltal således at den yderste for-løkke har $n/2$ iterationer. Hvilke af følgende er gyldige udsagn om køretiden $T(n)$ af Loops$(n)$? (Læg mærke til at udsagnene ikke er uafhængige --- hvis der gælder $f(n) = O(n)$ så gælder også $f(n) = O(n\lg n)$, osv.) Vælg ét eller flere korrekte svar og beskriv hvordan du kom frem til dem (både positive og negative svar).

generelt har 

\[
\sum_{i=1}^{\frac{n}{2}} (\frac{n}{2} - i) = \frac{n}{2} \cdot \frac{n}{2} - \sum_{i=1}^{\frac{n}{2}} i = \frac{n^2}{4} - \frac{\frac{n}{2}(\frac{n}{2}+1)}{2} = \frac{n^2}{4} - \frac{n(n+2)}{8} = \frac{n^2-n-2n}{8} = \frac{n^2-3n}{8}
\]

dvs $T(n) = \Theta(n^2) $

\begin{enumerate}
    \item $T(n) = O(n)$. forkert
    \item $T(n) = O(n\lg n)$. forkert
    \item $T(n) = O(n^2)$. korrekt
    \item $T(n) = \Omega(n)$. korrekt 
    \item $T(n) = \Omega(n\lg n)$. korrekt
    \item $T(n) = \Omega(n^2)$. korrekt
\end{enumerate}

\section{Invarianter}
Betragt følgende funktion, i pseudo-kode notationen fra CLRS, hvor input $A$ er en tabel af heltal og $n$ er længden af $A$.

\begin{verbatim}
Second-Smallest(A,n)
1  m = A[1]
2  r = ∞
3  for i = 2 to n
4      if A[i] < m
5          r = m
6          m = A[i]
7      elseif A[i] < r
8          r = A[i]
9  return r
\end{verbatim}

Hvilke udsagn gyldige løkke-invarianter ved starten af hver iteration i for-løkken? Vælg ét eller flere korrekte svar og beskriv hvordan du kom frem til dem (både positive og negative svar).

\begin{enumerate}
    \item $m = \min(A[1],\ldots,A[i-1])$. korrekt, m er det mindste element vi indtil videre ser i vores array
    \item $m = \min(A[1],\ldots,A[n])$. nej, da $\min(A[1],\ldots,A[n]) \leq \min(A[1],\ldots,A[i-1])$
    \item $r \geq m$. Korrekt. Dette kan bevises ved at betragte de to mulige cases i hver iteration:
    \begin{enumerate}
        \item Hvis $A[i] < m$: Vi sætter $r = m$ og $m = A[i]$, så efter opdateringen er $r$ (gamle $m$) større end $m$ (nye $A[i]$)
        \item Hvis $A[i] \geq m$: To undercases:
            \begin{itemize}
                \item Hvis $A[i] < r$: Vi sætter $r = A[i]$, men da $A[i] \geq m$ gælder stadig $r \geq m$
                \item Hvis $A[i] \geq r$: Ingen ændringer, så invarianten bevares
            \end{itemize}
    \end{enumerate}

    I alle tilfælde bevares invarianten $r \geq m$.
    \item $m \geq 0$. nej, m er det mindste element, og det mindste element kan godt være negativt
    \item Mindst 1 element i $A[1],\ldots,A[i-1]$ er mindre end eller lig med $r$. korrekt. da vi allerede har vist $r \geq m$ samt at der kan eksisterer elementer $x_1 \dots x_k = m \leq r$
    \item Højst 2 elementer i $A[1],\ldots,A[i-1]$ er mindre end eller lig med $r$. forkert.
    Dette kan modbevises med et modeksempel: Hvis $A = [1,1,1,2]$, så vil vi efter tre iterationer have $m=1$, $r=1$, 
    og dermed tre elementer der er mindre end eller lig med $r$.
\end{enumerate}

\section{Induktionsbeviser}
I denne opgave betragter vi regulære $n$-kanter som disse eksempler ($n = 4$ og $n = 6$):

Vi er interesserede i antal diagonaler, dvs. linier der forbinder to hjørner i $n$-kanterne og er helt indeholdt i figuren. En 3-kant har ingen diagonaler da alle hjørner er naboer, mens vi for $n = 4$ og $n = 6$ har henholdsvis 2 og 9 diagonaler.

En diagonal deler en $n$-kant i en $n_1$-kant og en $n_2$ kant med en fælles kant og hvor $n_1 + n_2 = n + 2$. For eksempel deler 4-kantens diagonal den i to 3-kanter.

\begin{enumerate}
    \item a) Bevis ved induktion at antal diagonaler $d_n$ i en $n$-kant er lig med $n(n-3)/2$ for $n \geq 3$. Der lægges vægt på, at argumentationen er klar og koncis.
\end{enumerate}

\begin{enumerate}
    \item \textbf{Base case:} For $n=3$, $d_3 = 0 = \frac{3(3-3)}{2}$, så formlen holder.
    
    \item \textbf{Induktionsskridt:} Antag at formlen holder for et vilkårligt $n \geq 3$, dvs.
    induktionshypotesen er at $d_n = \frac{n(n-3)}{2}$.
    
    Vi skal vise at formlen også holder for $n+1$, dvs. at $d_{n+1} = \frac{(n+1)(n-2)}{2}$.
    
    Betragt en regulær $(n+1)$-kant. Fra et vilkårligt hjørne $v$ kan vi tegne diagonaler til alle
    andre hjørner undtagen de to nabohjørner. Dette giver $(n+1-3) = n-2$ nye diagonaler fra $v$.
    
    De resterende diagonaler er præcis diagonalerne i den $n$-kant der fremkommer ved at fjerne
    hjørnet $v$. Efter induktionshypotesen er antallet af disse diagonaler $\frac{n(n-3)}{2}$.
    
    Derfor er det totale antal diagonaler i $(n+1)$-kanten:
    \begin{align*}
        d_{n+1} &= (n-2) + \frac{n(n-3)}{2} \\
        &= \frac{2(n-2) + n(n-3)}{2} \\
        &= \frac{2n-4 + n^2-3n}{2} \\
        &= \frac{n^2-n-4}{2} \\
        &= \frac{(n+1)(n-2)}{2}
    \end{align*}
\end{enumerate}


\section{Korrekthed og køretid}
I denne opgave bruger vi matrix og vektor-notation, se evt. CLRS appendix D.1. Fibonacci-tallene $F_0, F_1, F_2,\ldots$ er defineret ved $F_0 = 0$, $F_1 = 1$, og $F_n = F_{n-1} + F_{n-2}$ for $n > 1$. Betragt nu matricen $A$ og søjlevektorer på formen $x = (x_1, x_2)^T$.

\[
A = \begin{pmatrix} 0 & 1 \\ 1 & 1 \end{pmatrix} \qquad
x = \begin{pmatrix} x_1 \\ x_2 \end{pmatrix}
\]

\begin{enumerate}
    \item Argumentér for at for $x = (F_{n-2}, F_{n-1})^T$ opfylder matrix-vektor produktet ligheden $Ax = (F_{n-1}, F_n)^T$ for $n > 1$.
    
    \[
    A = \begin{pmatrix} 0 & 1 \\ 1 & 1 \end{pmatrix} (F_{n-2}, F_{n-1})^T \\ = 
    \\ \begin{pmatrix} F_{n-1} \\ F_{n-1}+F_{n-2} \end{pmatrix} = \\
    \begin{pmatrix} F_{n-1} \\ F_{n} \end{pmatrix} 
    \]
    
    \item Bevis at for $x = (0,1)^T$ gælder ligheden $A^{n-1}x = (F_{n-1}, F_n)^T$ for $n \geq 1$. 
    
    vi viser dette ved induktion.

    \begin{enumerate}
        \item base case n=2. 
        
        $A^{2-1}(0,1)^T = \begin{pmatrix} 0 & 1 \\ 1 & 1 \end{pmatrix}(0,1)^T = (1, 1) = (F_{1}, F_2) $

        \item For n+1:
        \begin{align*}
        A^{n}(0,1)^T &= A(A^{n-1}(0,1)^T) \\
        &= A(F_{n-1}, F_n)^T \text{ (by induction hypothesis)} \\
        &= \begin{pmatrix} 0 & 1 \\ 1 & 1 \end{pmatrix} \begin{pmatrix} F_{n-1} \\ F_n \end{pmatrix} \\
        &= \begin{pmatrix} F_n \\ F_{n-1} + F_n \end{pmatrix} \\
        &= \begin{pmatrix} F_n \\ F_{n+1} \end{pmatrix}
        \end{align*}

    \end{enumerate}

    
    \item Argumentér for at $A^{n-1}$, og dermed også $A^{n-1}x$, kan udregnes i tid $O(\lg n)$, hvis vi antager at multiplikation, addition og subtraktion tager konstant tid. (Start evt. med at se på tilfældet hvor $n-1$ er en potens af 2.)
    
    Vi kan beregne $A^n$ effektivt ved at bruge kvadrering:
    
    For eksempel, for at beregne $A^{13}$:
    \begin{align*}
        13 &= (1101)_2 \\
        A^{13} &= A^8 \cdot A^4 \cdot A^1
    \end{align*}
    
    Vi kan beregne dette ved at:
    \begin{enumerate}
        \item Beregn $A^2 = A \cdot A$
        \item Beregn $A^4 = A^2 \cdot A^2$
        \item Beregn $A^8 = A^4 \cdot A^4$
        \item Generelt $A^n = \prod_{log(n)} A^{\frac{n}{log(n)}}$
    \end{enumerate}
    
    Dette kræver kun $O(\lg n)$ matrix-multiplikationer i stedet for $n-1$ multiplikationer ved naiv tilgang.

    \item Argumentér for at $F_n$ kan udregnes i tid $O(\lg n)$ hvis vi antager at aritmetiske operationer tager konstant tid.

    Fra de foregående delspørgsmål har vi:
    \begin{enumerate}
        \item $A^{n-1}(0,1)^T = (F_{n-1}, F_n)^T$ (fra delspørgsmål b)
        \item $A^{n-1}$ kan beregnes i tid $O(\lg n)$ (fra delspørgsmål c)
    \end{enumerate}

    For at beregne $F_n$ kan vi derfor:
    \begin{enumerate}
        \item Beregne $A^{n-1}$ i tid $O(\lg n)$
        \item Multiplicere $A^{n-1}$ med vektoren $(0,1)^T$, hvilket tager konstant tid da $A$ er en $2 \times 2$ matrix
        \item Aflæse $F_n$ 
    \end{enumerate}

    Da hver af disse trin enten tager $O(\lg n)$ eller $O(1)$ tid, og vi udfører dem sekventielt, er den samlede køretid $O(\lg n)$.



\end{enumerate}





\end{document}


\bibliographystyle{plainnat}
\bibliography{references}