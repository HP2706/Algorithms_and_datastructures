\documentclass{article}
\usepackage{amsfonts}
\usepackage[utf8]{inputenc}
\usepackage[T1]{fontenc}
\usepackage{float}
\usepackage{bbm}
\usepackage[round]{natbib}
\usepackage{hyperref}
\usepackage{graphicx} % Required for inserting images
\usepackage{tikz}
\usepackage{parskip} % Add this to remove paragraph indentation
\usetikzlibrary{shapes,arrows,positioning,fit}
\usepackage{listings}
\usepackage{amsthm}  % Add this package for theorem-like environments
\usepackage{xcolor}
\usepackage{amsmath}
\usepackage{mathabx}
\usepackage{subcaption}  % For subfigures
\usepackage{algorithmicx}
\usepackage{algpseudocode}
\bibliographystyle{plain}

% Define theorem-like environments
\theoremstyle{definition}
\newtheorem{definition}{Definition}
\newtheorem{remark}{Remark}


\begin{document}

\section{Amortiseret analyse}

I denne opgave betragter vi potentialfunktionen der bruges til at analysere en binær tæller i CLRS sektion 16.3. Lad $\Phi(D_i)$ betegne potentialfunktionens værdi efter $i$ INCREMENT operationer.

\subsection*{a)} Hvilke af nedenstående egenskaber har $\Phi(D_i)$? (Læg mærke til at udsagnene ikke er uafhængige --- hvis der gælder $f(i) = \Omega(i)$ så gælder også $f(i) = \Omega(\lg i)$, osv.) Vælg ét eller flere korrekte svar og beskriv hvordan du kom frem til dem (både positive og negative svar).

\begin{enumerate}
    \item $\Phi(D_i) = \Omega(\lg i)$.
    \item $\Phi(D_i) = \Omega(i)$.
    \item $\Phi(D_i) = O(1)$.
    \item $\Phi(D_i) = O(\lg i)$.
    \item $\Phi(D_i) = O(i)$.
\end{enumerate}

Betragt nu situationen hvor vi gerne vil analysere en tæller i det almindelige base-10 talsystem. Tælleren skal repræsentere heltal med op til $k$ decimaler, startende med $00\ldots0$ (dvs. $k$ nuller) og understøtte en INCREMENT operation der øger tællerens værdi med 1 (op til den maksimale værdi $10^k - 1$). Lad $x_i$ betegne værdien af tælleren efter $i$ operationer (dvs. vi har $x_i = i$).

\subsection*{b)} Definér en potentialfunktion $\Phi(x_i)$ og brug den til at vise, at base-10 tælleren understøtter INCREMENT i amortiseret tid $O(1)$.

Antag nu at vi gerne også vil understøtte en DECREMENT operation der reducerer tællerens værdi med 1 (ned til den minimale værdi 0).

\subsection*{c)} Argumentér for at base-10 tælleren understøtter DECREMENT i amortiseret tid $O(k)$.

\subsection*{d)} Giv et eksempel på en sekvens af operationer som viser at det \textit{ikke} er muligt at give en konstant (uafhængig af $k$) grænse på den amortiserede omkostning for både INCREMENT og DECREMENT.

\section{Fibonacci hobe}

Betragt en Fibonacci hob $H$ (eng. \textit{Fibonacci heap}), som beskrevet i CLRS digitalt kapitel 19, hvorpå der udføres $n_1$ INSERT operationer, $n_2$ EXTRACT-MIN operationer, og $n_3$ DECREASE-KEY operationer. Det totale antal operationer er $n = n_1 + n_2 + n_3$. Hvilke af følgende udsagn er altid sande? Vælg ét eller flere korrekte svar og beskriv hvordan du kom frem til dem (både positive og negative svar).

\begin{enumerate}
    \item Den samlede worst case tid for operationerne er $\Omega(n\lg n)$.
    \item Den samlede worst case tid for operationerne er $O(n\lg n)$.
    \item Den samlede worst case tid for operationerne er $O(n_1 + n_3 + n_2\lg n)$.
    \item $H$ har $n_1 - n_2 - n_3$ knuder.
    \item $H$ har $m(H) = n_1 - n_2$ markerede knuder.
\end{enumerate}

\section{Korrekthedsargumenter}

Betragt følgende problem, kaldet MaxProduct:

Givet: En liste af tal $x_1,\ldots,x_n \in \mathbb{R}$ (kan være både positive og negative), find en delmængde $I^* \subseteq \{1,\ldots,n\}$ hvor produktet $\prod_{i\in I^*} x_i$ er så stort som muligt, dvs. for alle mængder $I \subseteq \{1,\ldots,n\}$ skal gælde at $\prod_{i\in I} x_i \leq \prod_{i\in I^*} x_i$.

\textbf{Eksempel.} På input $x_1 = -4$, $x_2 = -0.5$, $x_3 = 0.5$, $x_4 = 0.5$, $x_5 = 1.5$, $x_6 = 6$ kan vi vælge $I = \{5,6\}$ og få et produkt på $1.5 \cdot 6 = 9$, men et bedre valg er $I = \{1,2,5,6\}$ som giver produktet $(-4) \cdot (-0.5) \cdot 1.5 \cdot 6 = 18$. Den tomme mængde $I = \emptyset$ giver per definition $\prod_{i\in\emptyset} x_i = 1$.

Hvad gælder altid for en optimal løsning $I^*$? Vælg ét eller flere korrekte svar og beskriv hvordan du kom frem til dem (både positive og negative svar).

\begin{enumerate}
    \item $I^*$ indeholder \textit{ikke} noget indeks $i$ hvor $x_i$ er mellem 0 og 1 ($x_i \in (0,1)$).
    \item $I^*$ indeholder \textit{alle} indekser $i$ hvor $x_i < 0$.
    \item $I^*$ indeholder \textit{alle} indekser $i$ hvor $x_i > 1$.
    \item $I^*$ indeholder \textit{alle} indekser $i$ hvor $|x_i| > 1$ (i.e., absolutværdien af $x_i$ er større end 1).
\end{enumerate}

\section{Induktionsbeviser}

I denne opgave antager vi at $T$ er en sorteret tabel med $n > 1$ elementer således at for $i < j$ gælder $T[i] \leq T[j]$. Antag at $1 \leq p \leq r \leq n$, hvor $p$, $r$, $n$ er heltal. Betragt følgende funktion, i pseudo-kode notationen fra CLRS:

\begin{algorithmic}[1]
\Function{WhileMystery}{$x,T,p,r$}
    \State $low = p$
    \State $high = \max(p,r+1)$
    \While{$low < high$}
        \State $mid = \lfloor(low + high)/2\rfloor$
        \If{$x \leq T[mid]$}
            \State $high = mid$
        \Else
            \State $low = mid + 1$
        \EndIf
    \EndWhile
    \State \Return $high$
\EndFunction
\end{algorithmic}

\subsection*{a)} Lad $high_i$ og $low_i$ betegne værdien af variablene $high$ og $low$ efter $i$ iterationer af while løkken. Bevis ved induktion at $high_i - low_i \leq n/2^i$. (Du skal bruge antagelsen $1 \leq p \leq r \leq n$.) Der lægges vægt på, at argumentationen er klar og koncis.

\begin{enumerate}
    \item basecase:
    lad i = 0.
    da $1 \leq p \leq r \leq n$
    $high_{i=0} = r+1, low_{i=0} = p$ 
    $high_{i=0} - low_{i=0} = r+1 - p \leq r \leq n = \frac{n}{2^0}$ da $1 \leq p \leq r \leq n$
    \item 
    vi har induktionshypotesen: $high_n - low_n \leq n/2^n$ og undersøger $n+1$

    der er to cases
    \begin{enumerate}
        \item $x \leq T[mid_{n}]$
        \begin{align*}
            high_{i=n+1} &= mid_{n}, \quad low_{i=n+1}=low_{i=n} \\
            high_{i=n+1} - low_{i=n+1} &= mid_{n} - low_{i=n} \\
            &= \left\lfloor\frac{low_{i=n} + high_{i=n}}{2}\right\rfloor - low_{i=n} \\
            &\leq \frac{high_{i=n} - low_{i=n}}{2} \\
            &\leq \frac{n}{2^{n+1}} \quad \text{(ved induktionshypotesen)}
        \end{align*}

        \item $x > T[mid_{n}]$
        \begin{align*}
            high_{i=n+1} &= high_{i=n}, \quad low_{i=n+1}=mid_{n}+1 \\
            high_{i=n+1} - low_{i=n+1} &= high_{i=n} - (mid_{n}+1) \\
            &= high_{i=n} - \left(\left\lfloor\frac{low_{i=n} + high_{i=n}}{2}\right\rfloor+1\right) \\
            &\leq high_{i=n} - \frac{low_{i=n} + high_{i=n}}{2} \\
            &= \frac{high_{i=n} - low_{i=n}}{2} \\
            &\leq \frac{n}{2^{n+1}} \quad \text{(ved induktionshypotesen)}
        \end{align*}
    \end{enumerate}

    I begge tilfælde har vi vist at $high_{i=n+1} - low_{i=n+1} \leq n/2^{n+1}$, hvilket fuldfører induktionsbeviset.
\end{enumerate}


\end{document}


\bibliographystyle{plainnat}
\bibliography{references}