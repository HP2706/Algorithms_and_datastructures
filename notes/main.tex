\documentclass{article}
\usepackage{amsfonts}
\usepackage[utf8]{inputenc}
\usepackage[T1]{fontenc}
\usepackage{float}
\usepackage{bbm}
\usepackage[round]{natbib}
\usepackage{hyperref}
\usepackage{graphicx} % Required for inserting images
\usepackage{tikz}
\usepackage{parskip} % Add this to remove paragraph indentation
\usetikzlibrary{shapes,arrows,positioning,fit}
\usepackage{listings}
\usepackage{amsthm}  % Add this package for theorem-like environments
\usepackage{xcolor}
\usepackage{amsmath}
\usepackage{mathabx}
\usepackage{subcaption}  % For subfigures
\bibliographystyle{plain}

% Define theorem-like environments
\theoremstyle{definition}
\newtheorem{definition}{Definition}
\newtheorem{remark}{Remark}


\begin{document}

\section{Definitions}

\cite{devvrit2024matformernestedtransformerelastic}

\begin{remark}
We are mostly interested in functions $f$ from $\mathbb{N}$ to $\mathbb{R}_+ = [0,\infty)$.
\end{remark}

\begin{definition}[Big O]
Let $f,g : \mathbb{N} \to \mathbb{R}_+$. We write $f(n) \leq O(g(n))$ if there are constants $C,n_0 > 0$ so that for all $n \geq n_0$,
\[ f(n) \leq Cg(n). \]
\end{definition}

\begin{definition}[Big $\Omega$]
Let $f,g : \mathbb{N} \to \mathbb{R}_+$. We write $f(n) \geq \Omega(g(n))$ if there are constants $c,n_0 > 0$ so that for all $n \geq n_0$,
\[ f(n) \geq cg(n). \]
\end{definition}

\begin{remark}
Comparing Big O and Big $\Omega$ notation:
\begin{itemize}
    \item Big O provides an upper bound on the growth rate of a function
    \item Big $\Omega$ provides a lower bound on the growth rate of a function
    \item The constants $C$ and $c$ in the definitions represent scaling factors
    \item Both definitions require the inequality to hold for all values beyond some initial point $n_0$
    \item If a function is both $O(g(n))$ and $\Omega(g(n))$, we say it is $\Theta(g(n))$, meaning it grows at exactly the same rate as $g(n)$
\end{itemize}
\end{remark}

\begin{remark}
The following proof demonstrates the equivalence between $f(n) \leq O(g(n))$ and $g(n) = \Omega(f(n))$, showing these are just different ways of expressing the same relationship between functions.
\end{remark}

\begin{proof}
($\Rightarrow$) First, assume $f(n) \leq O(g(n))$. By definition, this means there exist constants $C, n_0 > 0$ such that for all $n \geq n_0$:
\[ f(n) \leq Cg(n) \]

Rearranging this inequality:
\[ g(n) \geq \frac{1}{C}f(n) \]

Let $c = \frac{1}{C}$. Since $C > 0$, we have $c > 0$. Therefore, there exists $c > 0$ and $n_0 > 0$ such that for all $n \geq n_0$:
\[ g(n) \geq cf(n) \]

This is precisely the definition of $g(n) = \Omega(f(n))$.

($\Leftarrow$) Conversely, assume $g(n) = \Omega(f(n))$. By definition, this means there exist constants $c, n_0 > 0$ such that for all $n \geq n_0$:
\[ g(n) \geq cf(n) \]

Rearranging:
\[ f(n) \leq \frac{1}{c}g(n) \]

Let $C = \frac{1}{c}$. Since $c > 0$, we have $C > 0$. Therefore, there exists $C > 0$ and $n_0 > 0$ such that for all $n \geq n_0$:
\[ f(n) \leq Cg(n) \]

This is precisely the definition of $f(n) \leq O(g(n))$.

Therefore, $f(n) \leq O(g(n))$ if and only if $g(n) = \Omega(f(n))$. \qed
\end{proof}

\section{Proofs}


    \begin{align}
        &\text{Proof by Strong Induction:} \notag \\[1em]
        &\text{Base case: } n = 2 \text{ is prime, so } n = \prod_{i=1}^1 p_i \text{ where } p_1 = 2. \notag \\[1em]
        &\text{Inductive Hypothesis: Assume the statement holds for all natural numbers } 2,3,\ldots,k. \notag \\[1em]
        &\text{Consider } k+1: \notag \\
        &\text{Case 1: If } k+1 \text{ is prime, then } k+1 = \prod_{i=1}^1 p_i \text{ where } p_1 = k+1. \notag \\[1em]
        &\text{Case 2: If } k+1 \text{ is composite, then } k+1 = a \cdot b \text{ where } 1 < a,b < k+1. \notag \\
        &\text{By the inductive hypothesis, } a = \prod_{i=1}^m p_i \text{ and } b = \prod_{j=1}^n q_j \text{ for some primes } p_i, q_j. \notag \\
        &\text{Therefore, } k+1 = a \cdot b = \left(\prod_{i=1}^m p_i\right) \cdot \left(\prod_{j=1}^n q_j\right) = \prod_{l=1}^{m+n} r_l \notag \\
        &\text{where } r_l \text{ represents the combined sequence of primes } p_i \text{ and } q_j. \notag \\[1em]
        &\text{By the principle of strong induction, the statement holds for all } n > 1. \quad \square \notag
    \end{align}
    


\end{document}


\bibliographystyle{plainnat}
\bibliography{references}